%% Generated by Sphinx.
\def\sphinxdocclass{report}
\documentclass[letterpaper,10pt,english]{sphinxmanual}
\ifdefined\pdfpxdimen
   \let\sphinxpxdimen\pdfpxdimen\else\newdimen\sphinxpxdimen
\fi \sphinxpxdimen=.75bp\relax

\usepackage[utf8]{inputenc}
\ifdefined\DeclareUnicodeCharacter
 \ifdefined\DeclareUnicodeCharacterAsOptional
  \DeclareUnicodeCharacter{"00A0}{\nobreakspace}
  \DeclareUnicodeCharacter{"2500}{\sphinxunichar{2500}}
  \DeclareUnicodeCharacter{"2502}{\sphinxunichar{2502}}
  \DeclareUnicodeCharacter{"2514}{\sphinxunichar{2514}}
  \DeclareUnicodeCharacter{"251C}{\sphinxunichar{251C}}
  \DeclareUnicodeCharacter{"2572}{\textbackslash}
 \else
  \DeclareUnicodeCharacter{00A0}{\nobreakspace}
  \DeclareUnicodeCharacter{2500}{\sphinxunichar{2500}}
  \DeclareUnicodeCharacter{2502}{\sphinxunichar{2502}}
  \DeclareUnicodeCharacter{2514}{\sphinxunichar{2514}}
  \DeclareUnicodeCharacter{251C}{\sphinxunichar{251C}}
  \DeclareUnicodeCharacter{2572}{\textbackslash}
 \fi
\fi
\usepackage{cmap}
\usepackage[T1]{fontenc}
\usepackage{amsmath,amssymb,amstext}
\usepackage{babel}
\usepackage{times}
\usepackage[Bjarne]{fncychap}
\usepackage[dontkeepoldnames]{sphinx}

\usepackage{geometry}

% Include hyperref last.
\usepackage{hyperref}
% Fix anchor placement for figures with captions.
\usepackage{hypcap}% it must be loaded after hyperref.
% Set up styles of URL: it should be placed after hyperref.
\urlstyle{same}

\addto\captionsenglish{\renewcommand{\figurename}{Fig.}}
\addto\captionsenglish{\renewcommand{\tablename}{Table}}
\addto\captionsenglish{\renewcommand{\literalblockname}{Listing}}

\addto\captionsenglish{\renewcommand{\literalblockcontinuedname}{continued from previous page}}
\addto\captionsenglish{\renewcommand{\literalblockcontinuesname}{continues on next page}}

\addto\extrasenglish{\def\pageautorefname{page}}

\setcounter{tocdepth}{0}



\title{MPPy Documentation}
\date{Dec 12, 2017}
\release{0.0.1}
\author{Tobias Machnitzki}
\newcommand{\sphinxlogo}{\vbox{}}
\renewcommand{\releasename}{Release}
\makeindex

\begin{document}

\maketitle
\sphinxtableofcontents
\phantomsection\label{\detokenize{index::doc}}


This package provides some tools for working with the data from the Barbados Cloud Observatory.


\chapter{First Steps}
\label{\detokenize{intro:first-steps}}\label{\detokenize{intro:documentation-of-the-mppy-project}}\label{\detokenize{intro::doc}}
In this section I will try to cover some basics to work with this module.


\section{Instruments}
\label{\detokenize{intro:instruments}}

\section{Tools}
\label{\detokenize{intro:tools}}

\chapter{Modules}
\label{\detokenize{modules:modules}}\label{\detokenize{modules::doc}}

\section{Instruments}
\label{\detokenize{modules:instruments}}

\subsection{Radar}
\label{\detokenize{MPPy.Instruments:radar}}\label{\detokenize{MPPy.Instruments:module-MPPy.Instruments.Radar}}\label{\detokenize{MPPy.Instruments::doc}}\index{MPPy.Instruments.Radar (module)}
This Module contains the Radar class. This class is for easy working with the BCO radar data.


\begin{savenotes}\sphinxatlongtablestart\begin{longtable}{p{0.5\linewidth}p{0.5\linewidth}}
\hline

\endfirsthead

\multicolumn{2}{c}%
{\makebox[0pt]{\sphinxtablecontinued{\tablename\ \thetable{} -- continued from previous page}}}\\
\hline

\endhead

\hline
\multicolumn{2}{r}{\makebox[0pt][r]{\sphinxtablecontinued{Continued on next page}}}\\
\endfoot

\endlastfoot

{\hyperref[\detokenize{generated/MPPy.Instruments.Radar.Radar:MPPy.Instruments.Radar.Radar}]{\sphinxcrossref{\sphinxcode{Radar}}}}(start, end{[}, device, version{]})
&
Class for working with radar data from Barbados.
\\
\hline
\end{longtable}\sphinxatlongtableend\end{savenotes}


\subsubsection{Radar class}
\label{\detokenize{generated/MPPy.Instruments.Radar.Radar:radar-class}}\label{\detokenize{generated/MPPy.Instruments.Radar.Radar::doc}}\index{Radar (class in MPPy.Instruments.Radar)}

\begin{fulllineitems}
\phantomsection\label{\detokenize{generated/MPPy.Instruments.Radar.Radar:MPPy.Instruments.Radar.Radar}}\pysiglinewithargsret{\sphinxbfcode{class }\sphinxcode{MPPy.Instruments.Radar.}\sphinxbfcode{Radar}}{\emph{start}, \emph{end}, \emph{device='CORAL'}, \emph{version=2}}{}
Class for working with radar data from Barbados.

Currently supported devices: CORAL, KATRIN
\begin{quote}\begin{description}
\item[{Parameters}] \leavevmode\begin{itemize}
\item {} 
\sphinxstyleliteralstrong{start} \textendash{} Either String or datetime.datetime-object indicating the start of the timefwindow

\item {} 
\sphinxstyleliteralstrong{end} \textendash{} Either String or datetime.datetime-object indicating the end of the timefwindow

\item {} 
\sphinxstyleliteralstrong{device} \textendash{} the device you want to use. Currently supported: CORAL, KATRIN

\item {} 
\sphinxstyleliteralstrong{version} \textendash{} The version of the dataset to use. Currently supported: 1,2,3  {[}note: 3 is in beta-phase{]}

\end{itemize}

\end{description}\end{quote}
\paragraph{Example}

The following example initiates a radar object for the CORAL with a timewindow form the 1st January 2017 to
the 2nd January 2017 to 3:30 pm:

\begin{sphinxVerbatim}[commandchars=\\\{\}]
\PYG{g+gp}{\PYGZgt{}\PYGZgt{}\PYGZgt{} }\PYG{n}{coral} \PYG{o}{=} \PYG{n}{Radar}\PYG{p}{(}\PYG{n}{start}\PYG{o}{=}\PYG{l+s+s2}{\PYGZdq{}}\PYG{l+s+s2}{20170101}\PYG{l+s+s2}{\PYGZdq{}}\PYG{p}{,}\PYG{n}{end}\PYG{o}{=}\PYG{l+s+s2}{\PYGZdq{}}\PYG{l+s+s2}{201701021530}\PYG{l+s+s2}{\PYGZdq{}}\PYG{p}{,} \PYG{n}{device}\PYG{o}{=}\PYG{l+s+s2}{\PYGZdq{}}\PYG{l+s+s2}{CORAL}\PYG{l+s+s2}{\PYGZdq{}}\PYG{p}{)}
\end{sphinxVerbatim}

To review the attributes of your class you can use:

\begin{sphinxVerbatim}[commandchars=\\\{\}]
\PYG{g+gp}{\PYGZgt{}\PYGZgt{}\PYGZgt{} }\PYG{n+nb}{print}\PYG{p}{(}\PYG{n}{coral}\PYG{p}{)}
\PYG{g+go}{CORAL Radar.}
\PYG{g+go}{Used data version 2.}
\PYG{g+go}{Load data from 2017\PYGZhy{}01\PYGZhy{}01 00:00:00 to 2017\PYGZhy{}01\PYGZhy{}02 15:30:00.}
\end{sphinxVerbatim}

To get attributes of the device you just need to call the attribute now:

\begin{sphinxVerbatim}[commandchars=\\\{\}]
\PYG{g+gp}{\PYGZgt{}\PYGZgt{}\PYGZgt{} }\PYG{n}{coral}\PYG{o}{.}\PYG{n}{lat}
\PYG{g+go}{array(13.162699699401855, dtype=float32)}
\end{sphinxVerbatim}

To get measured values you need to call the appropriate method:

\begin{sphinxVerbatim}[commandchars=\\\{\}]
\PYG{g+gp}{\PYGZgt{}\PYGZgt{}\PYGZgt{} }\PYG{n}{coral}\PYG{o}{.}\PYG{n}{getReflectivity}\PYG{p}{(}\PYG{n}{postprocessing}\PYG{o}{=}\PYG{l+s+s2}{\PYGZdq{}}\PYG{l+s+s2}{Zf}\PYG{l+s+s2}{\PYGZdq{}}\PYG{p}{)}
\PYG{g+go}{array([[...]], dtype=float32)}
\end{sphinxVerbatim}

In most cases you want the timestamp as well:

\begin{sphinxVerbatim}[commandchars=\\\{\}]
\PYG{g+gp}{\PYGZgt{}\PYGZgt{}\PYGZgt{} }\PYG{n}{coral}\PYG{o}{.}\PYG{n}{getTime}\PYG{p}{(}\PYG{p}{)}
\PYG{g+go}{array([datetime.datetime(2017, 1, 1, 1, 0, 18), ...,}
\PYG{g+go}{datetime.datetime(2017, 1, 2, 0, 59, 49)], dtype=object)}
\end{sphinxVerbatim}
\index{device (MPPy.Instruments.Radar.Radar attribute)}

\begin{fulllineitems}
\phantomsection\label{\detokenize{generated/MPPy.Instruments.Radar.Radar:MPPy.Instruments.Radar.Radar.device}}\pysigline{\sphinxbfcode{device}}
String of the device being used. (‘CORAL’ or ‘KATRIN’)

\end{fulllineitems}

\index{start (MPPy.Instruments.Radar.Radar attribute)}

\begin{fulllineitems}
\phantomsection\label{\detokenize{generated/MPPy.Instruments.Radar.Radar:MPPy.Instruments.Radar.Radar.start}}\pysigline{\sphinxbfcode{start}}
datetime.datetime object indicating the beginning of the chosen timewindow.

\end{fulllineitems}

\index{end (MPPy.Instruments.Radar.Radar attribute)}

\begin{fulllineitems}
\phantomsection\label{\detokenize{generated/MPPy.Instruments.Radar.Radar:MPPy.Instruments.Radar.Radar.end}}\pysigline{\sphinxbfcode{end}}
datetime.datetime object indicating the end of the chosen timewindow.

\end{fulllineitems}

\index{data\_version (MPPy.Instruments.Radar.Radar attribute)}

\begin{fulllineitems}
\phantomsection\label{\detokenize{generated/MPPy.Instruments.Radar.Radar:MPPy.Instruments.Radar.Radar.data_version}}\pysigline{\sphinxbfcode{data\_version}}
An Integer conatining the used version of the data (1,2,3{[}beta{]}) .

\end{fulllineitems}

\index{lat (MPPy.Instruments.Radar.Radar attribute)}

\begin{fulllineitems}
\phantomsection\label{\detokenize{generated/MPPy.Instruments.Radar.Radar:MPPy.Instruments.Radar.Radar.lat}}\pysigline{\sphinxbfcode{lat}}
Latitude of the instrument.

\end{fulllineitems}

\index{lon (MPPy.Instruments.Radar.Radar attribute)}

\begin{fulllineitems}
\phantomsection\label{\detokenize{generated/MPPy.Instruments.Radar.Radar:MPPy.Instruments.Radar.Radar.lon}}\pysigline{\sphinxbfcode{lon}}
Longitude of the instrument.

\end{fulllineitems}

\index{azimuth (MPPy.Instruments.Radar.Radar attribute)}

\begin{fulllineitems}
\phantomsection\label{\detokenize{generated/MPPy.Instruments.Radar.Radar:MPPy.Instruments.Radar.Radar.azimuth}}\pysigline{\sphinxbfcode{azimuth}}
Azimuth angle of where the instrument is pointing to.

\end{fulllineitems}

\index{elevation (MPPy.Instruments.Radar.Radar attribute)}

\begin{fulllineitems}
\phantomsection\label{\detokenize{generated/MPPy.Instruments.Radar.Radar:MPPy.Instruments.Radar.Radar.elevation}}\pysigline{\sphinxbfcode{elevation}}
Elevation angle of where the instrument is pointing to.

\end{fulllineitems}

\index{north (MPPy.Instruments.Radar.Radar attribute)}

\begin{fulllineitems}
\phantomsection\label{\detokenize{generated/MPPy.Instruments.Radar.Radar:MPPy.Instruments.Radar.Radar.north}}\pysigline{\sphinxbfcode{north}}
Degrees of where from the instrument seen is north.

\end{fulllineitems}

\paragraph{Methods}


\begin{savenotes}\sphinxatlongtablestart\begin{longtable}{p{0.5\linewidth}p{0.5\linewidth}}
\hline

\endfirsthead

\multicolumn{2}{c}%
{\makebox[0pt]{\sphinxtablecontinued{\tablename\ \thetable{} -- continued from previous page}}}\\
\hline

\endhead

\hline
\multicolumn{2}{r}{\makebox[0pt][r]{\sphinxtablecontinued{Continued on next page}}}\\
\endfoot

\endlastfoot

{\hyperref[\detokenize{generated/MPPy.Instruments.Radar.Radar.getTime:MPPy.Instruments.Radar.Radar.getTime}]{\sphinxcrossref{\sphinxcode{getTime}}}}()
&
Loads the time steps over the desired timeframe from all netCDF-files and returns them as one array.
\\
\hline
{\hyperref[\detokenize{generated/MPPy.Instruments.Radar.Radar.getRange:MPPy.Instruments.Radar.Radar.getRange}]{\sphinxcrossref{\sphinxcode{getRange}}}}()
&
Loads the range-gates from the netCDF-file which contains the last entries of the desired timeframe.
\\
\hline
{\hyperref[\detokenize{generated/MPPy.Instruments.Radar.Radar.getReflectivity:MPPy.Instruments.Radar.Radar.getReflectivity}]{\sphinxcrossref{\sphinxcode{getReflectivity}}}}({[}postprocessing{]})
&
Loads the reflecitivity over the desired timeframe from multiple netCDF-files and returns them as one array.
\\
\hline
{\hyperref[\detokenize{generated/MPPy.Instruments.Radar.Radar.getVelocity:MPPy.Instruments.Radar.Radar.getVelocity}]{\sphinxcrossref{\sphinxcode{getVelocity}}}}({[}target{]})
&
Loads the doppler velocity from the netCDF-files and returns them as one array
\\
\hline
{\hyperref[\detokenize{generated/MPPy.Instruments.Radar.Radar.getRadarConstant:MPPy.Instruments.Radar.Radar.getRadarConstant}]{\sphinxcrossref{\sphinxcode{getRadarConstant}}}}()
&
Loads the radar constant from all netCDF-Files and returns them as one array.
\\
\hline
{\hyperref[\detokenize{generated/MPPy.Instruments.Radar.Radar.getMeltHeight:MPPy.Instruments.Radar.Radar.getMeltHeight}]{\sphinxcrossref{\sphinxcode{getMeltHeight}}}}()
&
Loads the melting layer height from all netCDF-Files and returns them as one array.
\\
\hline
{\hyperref[\detokenize{generated/MPPy.Instruments.Radar.Radar.getNoisePower:MPPy.Instruments.Radar.Radar.getNoisePower}]{\sphinxcrossref{\sphinxcode{getNoisePower}}}}(channel)
&
Loads the HSdiv Noise Power in DSP of the desired channel from all netCDF-Files returns them as one array.
\\
\hline
{\hyperref[\detokenize{generated/MPPy.Instruments.Radar.Radar.getLDR:MPPy.Instruments.Radar.Radar.getLDR}]{\sphinxcrossref{\sphinxcode{getLDR}}}}({[}target{]})
&
Loads the linear depolarization ratio (LDR) in dbZ of the desired target from all netCDF-Files returns them as one  array.
\\
\hline
{\hyperref[\detokenize{generated/MPPy.Instruments.Radar.Radar.getRMS:MPPy.Instruments.Radar.Radar.getRMS}]{\sphinxcrossref{\sphinxcode{getRMS}}}}({[}target{]})
&
Loads the Peak Width RMS in m/s of the desired target from all netCDF-Files returns them as one  array.
\\
\hline
{\hyperref[\detokenize{generated/MPPy.Instruments.Radar.Radar.getSNR:MPPy.Instruments.Radar.Radar.getSNR}]{\sphinxcrossref{\sphinxcode{getSNR}}}}({[}target{]})
&
Loads the reflectivity SNR in dbZ of the desired target from all netCDF-Files returns them as one  array.
\\
\hline
{\hyperref[\detokenize{generated/MPPy.Instruments.Radar.Radar.getTransmitPower:MPPy.Instruments.Radar.Radar.getTransmitPower}]{\sphinxcrossref{\sphinxcode{getTransmitPower}}}}()
&
Loads the average transmit power in Watt of the desired target from all netCDF-Files returns them as one  array.
\\
\hline
{\hyperref[\detokenize{generated/MPPy.Instruments.Radar.Radar.quickplot2D:MPPy.Instruments.Radar.Radar.quickplot2D}]{\sphinxcrossref{\sphinxcode{quickplot2D}}}}(value{[}, save\_name, save\_path, ylim{]})
&
Creates a fast Quickplot from the input value.
\\
\hline
{\hyperref[\detokenize{generated/MPPy.Instruments.Radar.Radar.help:MPPy.Instruments.Radar.Radar.help}]{\sphinxcrossref{\sphinxcode{help}}}}()
&
This is a function for less experienced python-users.
\\
\hline
\end{longtable}\sphinxatlongtableend\end{savenotes}


\paragraph{getTime}
\label{\detokenize{generated/MPPy.Instruments.Radar.Radar.getTime:gettime}}\label{\detokenize{generated/MPPy.Instruments.Radar.Radar.getTime::doc}}\index{getTime() (MPPy.Instruments.Radar.Radar method)}

\begin{fulllineitems}
\phantomsection\label{\detokenize{generated/MPPy.Instruments.Radar.Radar.getTime:MPPy.Instruments.Radar.Radar.getTime}}\pysiglinewithargsret{\sphinxcode{Radar.}\sphinxbfcode{getTime}}{}{}
Loads the time steps over the desired timeframe from all netCDF-files and returns them as one array.
\begin{quote}\begin{description}
\item[{Returns}] \leavevmode
A numpy array containing datetime.datetime objects

\end{description}\end{quote}
\paragraph{Example}

Getting the time-stamps from an an already initiated radar object ‘coral’:

\begin{sphinxVerbatim}[commandchars=\\\{\}]
\PYG{g+gp}{\PYGZgt{}\PYGZgt{}\PYGZgt{} }\PYG{n}{coral}\PYG{o}{.}\PYG{n}{getTime}\PYG{p}{(}\PYG{p}{)}
\end{sphinxVerbatim}

\end{fulllineitems}



\paragraph{getRange}
\label{\detokenize{generated/MPPy.Instruments.Radar.Radar.getRange:getrange}}\label{\detokenize{generated/MPPy.Instruments.Radar.Radar.getRange::doc}}\index{getRange() (MPPy.Instruments.Radar.Radar method)}

\begin{fulllineitems}
\phantomsection\label{\detokenize{generated/MPPy.Instruments.Radar.Radar.getRange:MPPy.Instruments.Radar.Radar.getRange}}\pysiglinewithargsret{\sphinxcode{Radar.}\sphinxbfcode{getRange}}{}{}
Loads the range-gates from the netCDF-file which contains the last entries of the desired timeframe.
Note: just containing the range-gates from the first valid file of all used netCDF-files. If the range-gating
changes over the input-timewindow, then you might run into issues.
\begin{quote}\begin{description}
\item[{Returns}] \leavevmode
A numpy array with height in meters

\end{description}\end{quote}
\paragraph{Example}

Getting the range-gates of an already initiated radar object called ‘coral’:

\begin{sphinxVerbatim}[commandchars=\\\{\}]
\PYG{g+gp}{\PYGZgt{}\PYGZgt{}\PYGZgt{} }\PYG{n}{coral}\PYG{o}{.}\PYG{n}{getRange}\PYG{p}{(}\PYG{p}{)}
\end{sphinxVerbatim}

\end{fulllineitems}



\paragraph{getReflectivity}
\label{\detokenize{generated/MPPy.Instruments.Radar.Radar.getReflectivity:getreflectivity}}\label{\detokenize{generated/MPPy.Instruments.Radar.Radar.getReflectivity::doc}}\index{getReflectivity() (MPPy.Instruments.Radar.Radar method)}

\begin{fulllineitems}
\phantomsection\label{\detokenize{generated/MPPy.Instruments.Radar.Radar.getReflectivity:MPPy.Instruments.Radar.Radar.getReflectivity}}\pysiglinewithargsret{\sphinxcode{Radar.}\sphinxbfcode{getReflectivity}}{\emph{postprocessing='Zf'}}{}
Loads the reflecitivity over the desired timeframe from multiple netCDF-files and returns them as one array.
\begin{quote}\begin{description}
\item[{Parameters}] \leavevmode
\sphinxstyleliteralstrong{postprocessing} \textendash{} see Radar.help() for more inforamation

\item[{Returns}] \leavevmode
2-D numpy array with getReflectivity in dbz

\end{description}\end{quote}
\paragraph{Example}

Getting the unfiltered and mie corrected reflectivity of all hydrometeors with an an already
initiated radar object ‘coral’:

\begin{sphinxVerbatim}[commandchars=\\\{\}]
\PYG{g+gp}{\PYGZgt{}\PYGZgt{}\PYGZgt{} }\PYG{n}{coral}\PYG{o}{.}\PYG{n}{getReflectivity}\PYG{p}{(}\PYG{n}{postprocessing}\PYG{o}{=}\PYG{l+s+s2}{\PYGZdq{}}\PYG{l+s+s2}{Zu}\PYG{l+s+s2}{\PYGZdq{}}\PYG{p}{)}
\end{sphinxVerbatim}

\end{fulllineitems}



\paragraph{getVelocity}
\label{\detokenize{generated/MPPy.Instruments.Radar.Radar.getVelocity::doc}}\label{\detokenize{generated/MPPy.Instruments.Radar.Radar.getVelocity:getvelocity}}\index{getVelocity() (MPPy.Instruments.Radar.Radar method)}

\begin{fulllineitems}
\phantomsection\label{\detokenize{generated/MPPy.Instruments.Radar.Radar.getVelocity:MPPy.Instruments.Radar.Radar.getVelocity}}\pysiglinewithargsret{\sphinxcode{Radar.}\sphinxbfcode{getVelocity}}{\emph{target='hydrometeors'}}{}
Loads the doppler velocity from the netCDF-files and returns them as one array
\begin{quote}\begin{description}
\item[{Parameters}] \leavevmode
\sphinxstyleliteralstrong{target} \textendash{} String of which target the velocity you want to get from: ‘hydrometeors’ or ‘all’.

\item[{Returns}] \leavevmode
2-D numpy array with doppler velocity in m/s

\end{description}\end{quote}
\paragraph{Example}

This is how you could get the velocity from all targets for the 13th August 2016 to the 15th August 2016
of CORAL:

\begin{sphinxVerbatim}[commandchars=\\\{\}]
\PYG{g+gp}{\PYGZgt{}\PYGZgt{}\PYGZgt{} }\PYG{n}{coral} \PYG{o}{=} \PYG{n}{Radar}\PYG{p}{(}\PYG{n}{start}\PYG{o}{=}\PYG{l+s+s2}{\PYGZdq{}}\PYG{l+s+s2}{20160813}\PYG{l+s+s2}{\PYGZdq{}}\PYG{p}{,}\PYG{n}{end}\PYG{o}{=}\PYG{l+s+s2}{\PYGZdq{}}\PYG{l+s+s2}{20160815}\PYG{l+s+s2}{\PYGZdq{}}\PYG{p}{,} \PYG{n}{device}\PYG{o}{=}\PYG{l+s+s2}{\PYGZdq{}}\PYG{l+s+s2}{CORAL}\PYG{l+s+s2}{\PYGZdq{}}\PYG{p}{)}
\PYG{g+gp}{\PYGZgt{}\PYGZgt{}\PYGZgt{} }\PYG{n}{velocity} \PYG{o}{=} \PYG{n}{Radar}\PYG{o}{.}\PYG{n}{getVelocity}\PYG{p}{(}\PYG{n}{target}\PYG{o}{=}\PYG{l+s+s2}{\PYGZdq{}}\PYG{l+s+s2}{all}\PYG{l+s+s2}{\PYGZdq{}}\PYG{p}{)}
\end{sphinxVerbatim}

\end{fulllineitems}



\paragraph{getRadarConstant}
\label{\detokenize{generated/MPPy.Instruments.Radar.Radar.getRadarConstant:getradarconstant}}\label{\detokenize{generated/MPPy.Instruments.Radar.Radar.getRadarConstant::doc}}\index{getRadarConstant() (MPPy.Instruments.Radar.Radar method)}

\begin{fulllineitems}
\phantomsection\label{\detokenize{generated/MPPy.Instruments.Radar.Radar.getRadarConstant:MPPy.Instruments.Radar.Radar.getRadarConstant}}\pysiglinewithargsret{\sphinxcode{Radar.}\sphinxbfcode{getRadarConstant}}{}{}
Loads the radar constant from all netCDF-Files and returns them as one array.
\begin{quote}\begin{description}
\item[{Returns}] \leavevmode
A numpy array containing the radar constant in mm\textasciicircum{}6/m\textasciicircum{}3 for each timestep.

\end{description}\end{quote}

\end{fulllineitems}



\paragraph{getMeltHeight}
\label{\detokenize{generated/MPPy.Instruments.Radar.Radar.getMeltHeight:getmeltheight}}\label{\detokenize{generated/MPPy.Instruments.Radar.Radar.getMeltHeight::doc}}\index{getMeltHeight() (MPPy.Instruments.Radar.Radar method)}

\begin{fulllineitems}
\phantomsection\label{\detokenize{generated/MPPy.Instruments.Radar.Radar.getMeltHeight:MPPy.Instruments.Radar.Radar.getMeltHeight}}\pysiglinewithargsret{\sphinxcode{Radar.}\sphinxbfcode{getMeltHeight}}{}{}
Loads the melting layer height from all netCDF-Files and returns them as one array.
\begin{quote}\begin{description}
\item[{Returns}] \leavevmode
A numpy array containing the melting layer height in meters.

\end{description}\end{quote}

\end{fulllineitems}



\paragraph{getNoisePower}
\label{\detokenize{generated/MPPy.Instruments.Radar.Radar.getNoisePower::doc}}\label{\detokenize{generated/MPPy.Instruments.Radar.Radar.getNoisePower:getnoisepower}}\index{getNoisePower() (MPPy.Instruments.Radar.Radar method)}

\begin{fulllineitems}
\phantomsection\label{\detokenize{generated/MPPy.Instruments.Radar.Radar.getNoisePower:MPPy.Instruments.Radar.Radar.getNoisePower}}\pysiglinewithargsret{\sphinxcode{Radar.}\sphinxbfcode{getNoisePower}}{\emph{channel}}{}
Loads the HSdiv Noise Power in DSP of the desired channel from all netCDF-Files returns them as one array.
\begin{quote}\begin{description}
\item[{Parameters}] \leavevmode
\sphinxstyleliteralstrong{channel} \textendash{} String: can be either “Co” or “Cross”.

\item[{Returns}] \leavevmode
2D-numpy array containing the HSdiv Noise Power in DSP for all heigts and timesteps.

\end{description}\end{quote}

\end{fulllineitems}



\paragraph{getLDR}
\label{\detokenize{generated/MPPy.Instruments.Radar.Radar.getLDR:getldr}}\label{\detokenize{generated/MPPy.Instruments.Radar.Radar.getLDR::doc}}\index{getLDR() (MPPy.Instruments.Radar.Radar method)}

\begin{fulllineitems}
\phantomsection\label{\detokenize{generated/MPPy.Instruments.Radar.Radar.getLDR:MPPy.Instruments.Radar.Radar.getLDR}}\pysiglinewithargsret{\sphinxcode{Radar.}\sphinxbfcode{getLDR}}{\emph{target='hydrometeors'}}{}~\begin{description}
\item[{Loads the linear depolarization ratio (LDR) in dbZ of the desired target from all netCDF-Files returns them as one}] \leavevmode
array. Allowed targets are: “hydrometeors” or “all”. The default is “hydrometeors”.

\end{description}
\begin{quote}\begin{description}
\item[{Parameters}] \leavevmode
\sphinxstyleliteralstrong{target} \textendash{} String: can be either “hydrometeors” or “all”

\item[{Returns}] \leavevmode
2D-numpy array containing LDR in dbZ for all heigts and timesteps.

\end{description}\end{quote}

\end{fulllineitems}



\paragraph{getRMS}
\label{\detokenize{generated/MPPy.Instruments.Radar.Radar.getRMS:getrms}}\label{\detokenize{generated/MPPy.Instruments.Radar.Radar.getRMS::doc}}\index{getRMS() (MPPy.Instruments.Radar.Radar method)}

\begin{fulllineitems}
\phantomsection\label{\detokenize{generated/MPPy.Instruments.Radar.Radar.getRMS:MPPy.Instruments.Radar.Radar.getRMS}}\pysiglinewithargsret{\sphinxcode{Radar.}\sphinxbfcode{getRMS}}{\emph{target='hydrometeors'}}{}~\begin{description}
\item[{Loads the Peak Width RMS in m/s of the desired target from all netCDF-Files returns them as one}] \leavevmode
array. Allowed targets are: “hydrometeors” or “all”. The default is “hydrometeors”.

\end{description}
\begin{quote}\begin{description}
\item[{Parameters}] \leavevmode
\sphinxstyleliteralstrong{target} \textendash{} String: can be either “hydrometeors” or “all”

\item[{Returns}] \leavevmode
2D-numpy array containing LDR in m/s for all heigts and timesteps.

\end{description}\end{quote}

\end{fulllineitems}



\paragraph{getSNR}
\label{\detokenize{generated/MPPy.Instruments.Radar.Radar.getSNR:getsnr}}\label{\detokenize{generated/MPPy.Instruments.Radar.Radar.getSNR::doc}}\index{getSNR() (MPPy.Instruments.Radar.Radar method)}

\begin{fulllineitems}
\phantomsection\label{\detokenize{generated/MPPy.Instruments.Radar.Radar.getSNR:MPPy.Instruments.Radar.Radar.getSNR}}\pysiglinewithargsret{\sphinxcode{Radar.}\sphinxbfcode{getSNR}}{\emph{target='hydrometeors'}}{}~\begin{description}
\item[{Loads the reflectivity SNR in dbZ of the desired target from all netCDF-Files returns them as one}] \leavevmode
array. Allowed targets are: “hydrometeors”, “all” or “plank”. The default is “hydrometeors”.

\end{description}
\begin{quote}\begin{description}
\item[{Parameters}] \leavevmode
\sphinxstyleliteralstrong{target} \textendash{} String: can be either “hydrometeors”,”plank” or “all”

\item[{Returns}] \leavevmode
2D-numpy array containing LDR in dbZ for all heigts and timesteps.

\end{description}\end{quote}

\end{fulllineitems}



\paragraph{getTransmitPower}
\label{\detokenize{generated/MPPy.Instruments.Radar.Radar.getTransmitPower:gettransmitpower}}\label{\detokenize{generated/MPPy.Instruments.Radar.Radar.getTransmitPower::doc}}\index{getTransmitPower() (MPPy.Instruments.Radar.Radar method)}

\begin{fulllineitems}
\phantomsection\label{\detokenize{generated/MPPy.Instruments.Radar.Radar.getTransmitPower:MPPy.Instruments.Radar.Radar.getTransmitPower}}\pysiglinewithargsret{\sphinxcode{Radar.}\sphinxbfcode{getTransmitPower}}{}{}~\begin{quote}

Loads the average transmit power in Watt of the desired target from all netCDF-Files returns them as one
array.
\end{quote}
\begin{quote}\begin{description}
\item[{Returns}] \leavevmode
2D-numpy array containing average transmit power for all heigts and timesteps in W.

\end{description}\end{quote}

\end{fulllineitems}



\paragraph{quickplot2D}
\label{\detokenize{generated/MPPy.Instruments.Radar.Radar.quickplot2D:quickplot2d}}\label{\detokenize{generated/MPPy.Instruments.Radar.Radar.quickplot2D::doc}}\index{quickplot2D() (MPPy.Instruments.Radar.Radar method)}

\begin{fulllineitems}
\phantomsection\label{\detokenize{generated/MPPy.Instruments.Radar.Radar.quickplot2D:MPPy.Instruments.Radar.Radar.quickplot2D}}\pysiglinewithargsret{\sphinxcode{Radar.}\sphinxbfcode{quickplot2D}}{\emph{value}, \emph{save\_name=None}, \emph{save\_path=None}, \emph{ylim=None}}{}
Creates a fast Quickplot from the input value. Start and end date are the initialization-dates. To save the
picture you can provide a name for the picture (save\_name). If no savepath is provided, the picture will be
stored in the current working directory.
\begin{quote}\begin{description}
\item[{Parameters}] \leavevmode\begin{itemize}
\item {} 
\sphinxstyleliteralstrong{value} \textendash{} A 2-D array which you want to plot.

\item {} 
\sphinxstyleliteralstrong{save\_name} \textendash{} String: If provided picture will be saved under the given name. Example: ‘quicklook.png’

\item {} 
\sphinxstyleliteralstrong{save\_path} \textendash{} String: If provided, the picture will be saved at this location. Example: ‘/user/hoe/testuer/’

\item {} 
\sphinxstyleliteralstrong{ylim} \textendash{} Tuple: If provided the y-axis will be limited to these values.

\end{itemize}

\end{description}\end{quote}
\paragraph{Example}

To just get a quicklook of the reflectivity to your screen try:

\begin{sphinxVerbatim}[commandchars=\\\{\}]
\PYG{g+gp}{\PYGZgt{}\PYGZgt{}\PYGZgt{} }\PYG{n}{coral} \PYG{o}{=} \PYG{n}{Radar}\PYG{p}{(}\PYG{n}{start}\PYG{o}{=}\PYG{l+s+s2}{\PYGZdq{}}\PYG{l+s+s2}{2017040215}\PYG{l+s+s2}{\PYGZdq{}}\PYG{p}{,}\PYG{n}{end}\PYG{o}{=}\PYG{l+s+s2}{\PYGZdq{}}\PYG{l+s+s2}{201704021530}\PYG{l+s+s2}{\PYGZdq{}}\PYG{p}{,} \PYG{n}{device}\PYG{o}{=}\PYG{l+s+s2}{\PYGZdq{}}\PYG{l+s+s2}{CORAL}\PYG{l+s+s2}{\PYGZdq{}}\PYG{p}{)}
\PYG{g+gp}{\PYGZgt{}\PYGZgt{}\PYGZgt{} }\PYG{n}{coral}\PYG{o}{.}\PYG{n}{quickplot2D}\PYG{p}{(}\PYG{n}{value}\PYG{o}{=}\PYG{n}{coral}\PYG{o}{.}\PYG{n}{getReflectivity}\PYG{p}{(}\PYG{p}{)}\PYG{p}{,}\PYG{n}{ylim}\PYG{o}{=}\PYG{p}{(}\PYG{l+m+mi}{100}\PYG{p}{,}\PYG{l+m+mi}{2000}\PYG{p}{)}\PYG{p}{)}
\end{sphinxVerbatim}

\end{fulllineitems}



\paragraph{help}
\label{\detokenize{generated/MPPy.Instruments.Radar.Radar.help:help}}\label{\detokenize{generated/MPPy.Instruments.Radar.Radar.help::doc}}\index{help() (MPPy.Instruments.Radar.Radar static method)}

\begin{fulllineitems}
\phantomsection\label{\detokenize{generated/MPPy.Instruments.Radar.Radar.help:MPPy.Instruments.Radar.Radar.help}}\pysiglinewithargsret{\sphinxbfcode{static }\sphinxcode{Radar.}\sphinxbfcode{help}}{}{}
This is a function for less experienced python-users. It will print some tipps for working with this Radar
class. If possible use the documentation, it will be much more likely up to date and contains more information!
\begin{quote}\begin{description}
\item[{Returns}] \leavevmode
Just prints some help messages into the console

\end{description}\end{quote}

\end{fulllineitems}


\end{fulllineitems}



\section{Tools}
\label{\detokenize{modules:tools}}

\subsection{Tools}
\label{\detokenize{MPPy.tools:module-MPPy.tools.tools}}\label{\detokenize{MPPy.tools:tools}}\label{\detokenize{MPPy.tools::doc}}\index{MPPy.tools.tools (module)}
This toolbox contains some functions which are being used by the MPPy package but might be usefull to the
enduser, as well.


\begin{savenotes}\sphinxatlongtablestart\begin{longtable}{p{0.5\linewidth}p{0.5\linewidth}}
\hline

\endfirsthead

\multicolumn{2}{c}%
{\makebox[0pt]{\sphinxtablecontinued{\tablename\ \thetable{} -- continued from previous page}}}\\
\hline

\endhead

\hline
\multicolumn{2}{r}{\makebox[0pt][r]{\sphinxtablecontinued{Continued on next page}}}\\
\endfoot

\endlastfoot

{\hyperref[\detokenize{generated/MPPy.tools.tools.daterange:MPPy.tools.tools.daterange}]{\sphinxcrossref{\sphinxcode{daterange}}}}(start\_date, end\_date)
&
This function is for looping over datetime.datetime objects within a timeframe from start\_date to end\_date.
\\
\hline
{\hyperref[\detokenize{generated/MPPy.tools.tools.num2time:MPPy.tools.tools.num2time}]{\sphinxcrossref{\sphinxcode{num2time}}}}(num)
&
Converts seconds since 1970 to datetime objects.
\\
\hline
{\hyperref[\detokenize{generated/MPPy.tools.tools.time2num:MPPy.tools.tools.time2num}]{\sphinxcrossref{\sphinxcode{time2num}}}}(time)
&
Converts a datetime.datetime object to seconds since 1970 as float.
\\
\hline
{\hyperref[\detokenize{generated/MPPy.tools.tools.datestr:MPPy.tools.tools.datestr}]{\sphinxcrossref{\sphinxcode{datestr}}}}(dt\_obj)
&
Converts a datetime.datetime object to a string in the commonly used shape for this module.
\\
\hline
\end{longtable}\sphinxatlongtableend\end{savenotes}


\subsubsection{daterange}
\label{\detokenize{generated/MPPy.tools.tools.daterange::doc}}\label{\detokenize{generated/MPPy.tools.tools.daterange:daterange}}\index{daterange() (in module MPPy.tools.tools)}

\begin{fulllineitems}
\phantomsection\label{\detokenize{generated/MPPy.tools.tools.daterange:MPPy.tools.tools.daterange}}\pysiglinewithargsret{\sphinxcode{MPPy.tools.tools.}\sphinxbfcode{daterange}}{\emph{start\_date}, \emph{end\_date}}{}
This function is for looping over datetime.datetime objects within a timeframe from start\_date to end\_date.
It will only loop over days.
\begin{quote}\begin{description}
\item[{Parameters}] \leavevmode\begin{itemize}
\item {} 
\sphinxstyleliteralstrong{start\_date} \textendash{} datetime.datetime object

\item {} 
\sphinxstyleliteralstrong{end\_date} \textendash{} datetime.datetime object

\end{itemize}

\item[{Yields}] \leavevmode
A datetime.datetime object starting from start\_date and going to end\_date

\end{description}\end{quote}
\paragraph{Example}

If you want to loop over the dates from the 1st January 2017 to the 3rd January 2017:

\begin{sphinxVerbatim}[commandchars=\\\{\}]
\PYG{g+gp}{\PYGZgt{}\PYGZgt{}\PYGZgt{} }\PYG{n}{start} \PYG{o}{=} \PYG{n}{datetime}\PYG{o}{.}\PYG{n}{datetime}\PYG{p}{(}\PYG{l+m+mi}{2017}\PYG{p}{,}\PYG{l+m+mi}{1}\PYG{p}{,}\PYG{l+m+mi}{1}\PYG{p}{)}
\PYG{g+gp}{\PYGZgt{}\PYGZgt{}\PYGZgt{} }\PYG{n}{end} \PYG{o}{=} \PYG{n}{datetime}\PYG{o}{.}\PYG{n}{datetime}\PYG{p}{(}\PYG{l+m+mi}{2017}\PYG{p}{,}\PYG{l+m+mi}{1}\PYG{p}{,}\PYG{l+m+mi}{3}\PYG{p}{)}
\PYG{g+gp}{\PYGZgt{}\PYGZgt{}\PYGZgt{} }\PYG{k}{for} \PYG{n}{x} \PYG{o+ow}{in} \PYG{n}{daterange}\PYG{p}{(}\PYG{n}{start}\PYG{p}{,}\PYG{n}{end}\PYG{p}{)}\PYG{p}{:}
\PYG{g+gp}{\PYGZgt{}\PYGZgt{}\PYGZgt{} }    \PYG{n+nb}{print}\PYG{p}{(}\PYG{n+nb}{str}\PYG{p}{(}\PYG{n}{x}\PYG{p}{)}\PYG{p}{)}
\PYG{g+go}{2017\PYGZhy{}01\PYGZhy{}01 00:00:00}
\PYG{g+go}{2017\PYGZhy{}01\PYGZhy{}02 00:00:00}
\PYG{g+go}{2017\PYGZhy{}01\PYGZhy{}03 00:00:00}
\end{sphinxVerbatim}

\end{fulllineitems}



\subsubsection{num2time}
\label{\detokenize{generated/MPPy.tools.tools.num2time:num2time}}\label{\detokenize{generated/MPPy.tools.tools.num2time::doc}}\index{num2time() (in module MPPy.tools.tools)}

\begin{fulllineitems}
\phantomsection\label{\detokenize{generated/MPPy.tools.tools.num2time:MPPy.tools.tools.num2time}}\pysiglinewithargsret{\sphinxcode{MPPy.tools.tools.}\sphinxbfcode{num2time}}{\emph{num}}{}
Converts seconds since 1970 to datetime objects.
If input is a numpy array, ouput will be a numpy array as well.
\begin{quote}\begin{description}
\item[{Parameters}] \leavevmode
\sphinxstyleliteralstrong{num} \textendash{} float/ndarray.  seconds since 1970

\item[{Returns}] \leavevmode
datetime.datetime object

\end{description}\end{quote}

\end{fulllineitems}



\subsubsection{time2num}
\label{\detokenize{generated/MPPy.tools.tools.time2num:time2num}}\label{\detokenize{generated/MPPy.tools.tools.time2num::doc}}\index{time2num() (in module MPPy.tools.tools)}

\begin{fulllineitems}
\phantomsection\label{\detokenize{generated/MPPy.tools.tools.time2num:MPPy.tools.tools.time2num}}\pysiglinewithargsret{\sphinxcode{MPPy.tools.tools.}\sphinxbfcode{time2num}}{\emph{time}}{}
Converts a datetime.datetime object to seconds since 1970 as float.
If input is a numpy array, ouput will be a numpy array as well.
\begin{quote}\begin{description}
\item[{Parameters}] \leavevmode
\sphinxstyleliteralstrong{time} \textendash{} datetime.datetime object / ndarray of datetime.datetime objects.

\item[{Returns}] \leavevmode
Float of seconds since 1970 / ndarray of floats.

\end{description}\end{quote}

\end{fulllineitems}



\subsubsection{datestr}
\label{\detokenize{generated/MPPy.tools.tools.datestr:datestr}}\label{\detokenize{generated/MPPy.tools.tools.datestr::doc}}\index{datestr() (in module MPPy.tools.tools)}

\begin{fulllineitems}
\phantomsection\label{\detokenize{generated/MPPy.tools.tools.datestr:MPPy.tools.tools.datestr}}\pysiglinewithargsret{\sphinxcode{MPPy.tools.tools.}\sphinxbfcode{datestr}}{\emph{dt\_obj}}{}
Converts a datetime.datetime object to a string in the commonly used shape for this module.
\begin{quote}\begin{description}
\item[{Parameters}] \leavevmode
\sphinxstyleliteralstrong{dt\_obj} \textendash{} datetime.datetime object.

\item[{Returns}] \leavevmode
String of the format YYMMDD. Y=Year, M=Month, D=Day.

\end{description}\end{quote}

\end{fulllineitems}



\chapter{HELP}
\label{\detokenize{help:help}}\label{\detokenize{help::doc}}
If you need any help or have trouble with the project please contact \sphinxhref{mailto:tobias.machnitzki@mpimet.mpg.de}{tobias.machnitzki@mpimet.mpg.de}


\renewcommand{\indexname}{Python Module Index}
\begin{sphinxtheindex}
\def\bigletter#1{{\Large\sffamily#1}\nopagebreak\vspace{1mm}}
\bigletter{m}
\item {\sphinxstyleindexentry{MPPy.Instruments.Radar}}\sphinxstyleindexpageref{MPPy.Instruments:\detokenize{module-MPPy.Instruments.Radar}}
\item {\sphinxstyleindexentry{MPPy.tools.tools}}\sphinxstyleindexpageref{MPPy.tools:\detokenize{module-MPPy.tools.tools}}
\end{sphinxtheindex}

\renewcommand{\indexname}{Index}
\printindex
\end{document}